\documentclass[12pt,]{article}
\usepackage{lmodern}
\usepackage{amssymb,amsmath}
\usepackage{ifxetex,ifluatex}
\usepackage{fixltx2e} % provides \textsubscript
\ifnum 0\ifxetex 1\fi\ifluatex 1\fi=0 % if pdftex
  \usepackage[T1]{fontenc}
  \usepackage[utf8]{inputenc}
\else % if luatex or xelatex
  \ifxetex
    \usepackage{mathspec}
  \else
    \usepackage{fontspec}
  \fi
  \defaultfontfeatures{Ligatures=TeX,Scale=MatchLowercase}
\fi
% use upquote if available, for straight quotes in verbatim environments
\IfFileExists{upquote.sty}{\usepackage{upquote}}{}
% use microtype if available
\IfFileExists{microtype.sty}{%
\usepackage{microtype}
\UseMicrotypeSet[protrusion]{basicmath} % disable protrusion for tt fonts
}{}
\usepackage[margin=1in]{geometry}
\usepackage{hyperref}
\hypersetup{unicode=true,
            pdftitle={MEM5220 - R Econometrics Evaluation},
            pdfauthor={YOUR NAME HERE},
            pdfborder={0 0 0},
            breaklinks=true}
\urlstyle{same}  % don't use monospace font for urls
\usepackage{color}
\usepackage{fancyvrb}
\newcommand{\VerbBar}{|}
\newcommand{\VERB}{\Verb[commandchars=\\\{\}]}
\DefineVerbatimEnvironment{Highlighting}{Verbatim}{commandchars=\\\{\}}
% Add ',fontsize=\small' for more characters per line
\usepackage{framed}
\definecolor{shadecolor}{RGB}{248,248,248}
\newenvironment{Shaded}{\begin{snugshade}}{\end{snugshade}}
\newcommand{\AlertTok}[1]{\textcolor[rgb]{0.94,0.16,0.16}{#1}}
\newcommand{\AnnotationTok}[1]{\textcolor[rgb]{0.56,0.35,0.01}{\textbf{\textit{#1}}}}
\newcommand{\AttributeTok}[1]{\textcolor[rgb]{0.77,0.63,0.00}{#1}}
\newcommand{\BaseNTok}[1]{\textcolor[rgb]{0.00,0.00,0.81}{#1}}
\newcommand{\BuiltInTok}[1]{#1}
\newcommand{\CharTok}[1]{\textcolor[rgb]{0.31,0.60,0.02}{#1}}
\newcommand{\CommentTok}[1]{\textcolor[rgb]{0.56,0.35,0.01}{\textit{#1}}}
\newcommand{\CommentVarTok}[1]{\textcolor[rgb]{0.56,0.35,0.01}{\textbf{\textit{#1}}}}
\newcommand{\ConstantTok}[1]{\textcolor[rgb]{0.00,0.00,0.00}{#1}}
\newcommand{\ControlFlowTok}[1]{\textcolor[rgb]{0.13,0.29,0.53}{\textbf{#1}}}
\newcommand{\DataTypeTok}[1]{\textcolor[rgb]{0.13,0.29,0.53}{#1}}
\newcommand{\DecValTok}[1]{\textcolor[rgb]{0.00,0.00,0.81}{#1}}
\newcommand{\DocumentationTok}[1]{\textcolor[rgb]{0.56,0.35,0.01}{\textbf{\textit{#1}}}}
\newcommand{\ErrorTok}[1]{\textcolor[rgb]{0.64,0.00,0.00}{\textbf{#1}}}
\newcommand{\ExtensionTok}[1]{#1}
\newcommand{\FloatTok}[1]{\textcolor[rgb]{0.00,0.00,0.81}{#1}}
\newcommand{\FunctionTok}[1]{\textcolor[rgb]{0.00,0.00,0.00}{#1}}
\newcommand{\ImportTok}[1]{#1}
\newcommand{\InformationTok}[1]{\textcolor[rgb]{0.56,0.35,0.01}{\textbf{\textit{#1}}}}
\newcommand{\KeywordTok}[1]{\textcolor[rgb]{0.13,0.29,0.53}{\textbf{#1}}}
\newcommand{\NormalTok}[1]{#1}
\newcommand{\OperatorTok}[1]{\textcolor[rgb]{0.81,0.36,0.00}{\textbf{#1}}}
\newcommand{\OtherTok}[1]{\textcolor[rgb]{0.56,0.35,0.01}{#1}}
\newcommand{\PreprocessorTok}[1]{\textcolor[rgb]{0.56,0.35,0.01}{\textit{#1}}}
\newcommand{\RegionMarkerTok}[1]{#1}
\newcommand{\SpecialCharTok}[1]{\textcolor[rgb]{0.00,0.00,0.00}{#1}}
\newcommand{\SpecialStringTok}[1]{\textcolor[rgb]{0.31,0.60,0.02}{#1}}
\newcommand{\StringTok}[1]{\textcolor[rgb]{0.31,0.60,0.02}{#1}}
\newcommand{\VariableTok}[1]{\textcolor[rgb]{0.00,0.00,0.00}{#1}}
\newcommand{\VerbatimStringTok}[1]{\textcolor[rgb]{0.31,0.60,0.02}{#1}}
\newcommand{\WarningTok}[1]{\textcolor[rgb]{0.56,0.35,0.01}{\textbf{\textit{#1}}}}
\usepackage{graphicx,grffile}
\makeatletter
\def\maxwidth{\ifdim\Gin@nat@width>\linewidth\linewidth\else\Gin@nat@width\fi}
\def\maxheight{\ifdim\Gin@nat@height>\textheight\textheight\else\Gin@nat@height\fi}
\makeatother
% Scale images if necessary, so that they will not overflow the page
% margins by default, and it is still possible to overwrite the defaults
% using explicit options in \includegraphics[width, height, ...]{}
\setkeys{Gin}{width=\maxwidth,height=\maxheight,keepaspectratio}
\IfFileExists{parskip.sty}{%
\usepackage{parskip}
}{% else
\setlength{\parindent}{0pt}
\setlength{\parskip}{6pt plus 2pt minus 1pt}
}
\setlength{\emergencystretch}{3em}  % prevent overfull lines
\providecommand{\tightlist}{%
  \setlength{\itemsep}{0pt}\setlength{\parskip}{0pt}}
\setcounter{secnumdepth}{0}
% Redefines (sub)paragraphs to behave more like sections
\ifx\paragraph\undefined\else
\let\oldparagraph\paragraph
\renewcommand{\paragraph}[1]{\oldparagraph{#1}\mbox{}}
\fi
\ifx\subparagraph\undefined\else
\let\oldsubparagraph\subparagraph
\renewcommand{\subparagraph}[1]{\oldsubparagraph{#1}\mbox{}}
\fi

%%% Use protect on footnotes to avoid problems with footnotes in titles
\let\rmarkdownfootnote\footnote%
\def\footnote{\protect\rmarkdownfootnote}

%%% Change title format to be more compact
\usepackage{titling}

% Create subtitle command for use in maketitle
\newcommand{\subtitle}[1]{
  \posttitle{
    \begin{center}\large#1\end{center}
    }
}

\setlength{\droptitle}{-2em}

  \title{MEM5220 - R Econometrics Evaluation}
    \pretitle{\vspace{\droptitle}\centering\huge}
  \posttitle{\par}
    \author{YOUR NAME HERE}
    \preauthor{\centering\large\emph}
  \postauthor{\par}
    \date{}
    \predate{}\postdate{}
  
\usepackage{array}
\usepackage{caption}
\usepackage{graphicx}
\usepackage{siunitx}
\usepackage[table]{xcolor}
\usepackage{multirow}
\usepackage{hhline}
\usepackage{calc}
\usepackage{tabularx}

\begin{document}
\maketitle

\begin{center}\rule{0.5\linewidth}{\linethickness}\end{center}

This first R econometrics self-evaluation assignment is focused on data
cleaning, data manipulation, plotting and estimating and interpreting
simple linear regression models. You can use \textbf{any} additional
packages for answering the questions.

Packages I have used to solve the exercises:

\begin{Shaded}
\begin{Highlighting}[]
\KeywordTok{library}\NormalTok{(tidyverse)}
\KeywordTok{library}\NormalTok{(stargazer)}
\KeywordTok{library}\NormalTok{(huxtable)}
\KeywordTok{library}\NormalTok{(broom)}
\KeywordTok{library}\NormalTok{(lmtest)}
\KeywordTok{library}\NormalTok{(sandwich)}
\KeywordTok{library}\NormalTok{(car)}
\KeywordTok{library}\NormalTok{(magrittr)}
\end{Highlighting}
\end{Shaded}

\textbf{Note:}

\begin{itemize}
\tightlist
\item
  This assignemtn has to be solved in this R markdown and you should be
  able to ``knit'' the document without errors
\item
  Fill out your name in ``yaml'' - block on top of this document
\item
  Use the R markdown syntax:
\item
  Write your code in code chunks
\item
  Write your explanations including the equations in markdown syntax
\item
  If you have an error in your code use \texttt{\#} to comment the line
  out where the error occurs but do not delete the code itself
\end{itemize}

\begin{center}\rule{0.5\linewidth}{\linethickness}\end{center}

You will be working with the dataset \texttt{Caschool} from the
\textbf{Ecdat} package, the dataset \texttt{Wage1} from
\textbf{wooldrige} package. In the last step, you will be working with a
simulated dataset.

\hypertarget{caschool-exercises}{%
\section{Caschool exercises}\label{caschool-exercises}}

\hypertarget{question}{%
\subsection{Question:}\label{question}}

Load the dataset \texttt{Caschool} from the \textbf{Ecdat} package.

The Caschooldataset contains the average test scores of 420 elementary
schools in California along with some additional information.

\begin{Shaded}
\begin{Highlighting}[]
\CommentTok{# install.packages("Ecdat")}
\KeywordTok{library}\NormalTok{(}\StringTok{"Ecdat"}\NormalTok{)}
\KeywordTok{data}\NormalTok{(}\StringTok{"Caschool"}\NormalTok{, }\DataTypeTok{package =} \StringTok{"Ecdat"}\NormalTok{)}
\end{Highlighting}
\end{Shaded}

What are the dimensions of the \texttt{Caschool} dataset?

\textbf{A}:

\begin{Shaded}
\begin{Highlighting}[]
\KeywordTok{dim}\NormalTok{(Caschool)}
\end{Highlighting}
\end{Shaded}

\begin{verbatim}
[1] 420  17
\end{verbatim}

\hypertarget{question-1}{%
\subsection{Question:}\label{question-1}}

Display the structure of the Caschool dataset. Which variable has values
encoded as factors?

\textbf{A}:

\begin{Shaded}
\begin{Highlighting}[]
\KeywordTok{str}\NormalTok{(Caschool)}
\end{Highlighting}
\end{Shaded}

\begin{verbatim}
'data.frame':   420 obs. of  17 variables:
 $ distcod : int  75119 61499 61549 61457 61523 62042 68536 63834 62331 67306 ...
 $ county  : Factor w/ 45 levels "Alameda","Butte",..: 1 2 2 2 2 6 29 11 6 25 ...
 $ district: Factor w/ 409 levels "Ackerman Elementary",..: 362 214 367 132 270 53 152 383 263 94 ...
 $ grspan  : Factor w/ 2 levels "KK-06","KK-08": 2 2 2 2 2 2 2 2 2 1 ...
 $ enrltot : int  195 240 1550 243 1335 137 195 888 379 2247 ...
 $ teachers: num  10.9 11.1 82.9 14 71.5 ...
 $ calwpct : num  0.51 15.42 55.03 36.48 33.11 ...
 $ mealpct : num  2.04 47.92 76.32 77.05 78.43 ...
 $ computer: int  67 101 169 85 171 25 28 66 35 0 ...
 $ testscr : num  691 661 644 648 641 ...
 $ compstu : num  0.344 0.421 0.109 0.35 0.128 ...
 $ expnstu : num  6385 5099 5502 7102 5236 ...
 $ str     : num  17.9 21.5 18.7 17.4 18.7 ...
 $ avginc  : num  22.69 9.82 8.98 8.98 9.08 ...
 $ elpct   : num  0 4.58 30 0 13.86 ...
 $ readscr : num  692 660 636 652 642 ...
 $ mathscr : num  690 662 651 644 640 ...
\end{verbatim}

County, district and grspan are encoded as factors.

\hypertarget{question-2}{%
\subsection{Question:}\label{question-2}}

Provide a summary statistic of the data

\textbf{A}:

\begin{Shaded}
\begin{Highlighting}[]
\KeywordTok{summary}\NormalTok{(Caschool)}
\end{Highlighting}
\end{Shaded}

\begin{verbatim}
    distcod              county                         district  
 Min.   :61382   Sonoma     : 29   Lakeside Union Elementary:  3  
 1st Qu.:64308   Kern       : 27   Mountain View Elementary :  3  
 Median :67760   Los Angeles: 27   Jefferson Elementary     :  2  
 Mean   :67473   Tulare     : 24   Liberty Elementary       :  2  
 3rd Qu.:70419   San Diego  : 21   Ocean View Elementary    :  2  
 Max.   :75440   Santa Clara: 20   Pacific Union Elementary :  2  
                 (Other)    :272   (Other)                  :406  
   grspan       enrltot           teachers          calwpct      
 KK-06: 61   Min.   :   81.0   Min.   :   4.85   Min.   : 0.000  
 KK-08:359   1st Qu.:  379.0   1st Qu.:  19.66   1st Qu.: 4.395  
             Median :  950.5   Median :  48.56   Median :10.520  
             Mean   : 2628.8   Mean   : 129.07   Mean   :13.246  
             3rd Qu.: 3008.0   3rd Qu.: 146.35   3rd Qu.:18.981  
             Max.   :27176.0   Max.   :1429.00   Max.   :78.994  
                                                                 
    mealpct          computer         testscr         compstu       
 Min.   :  0.00   Min.   :   0.0   Min.   :605.5   Min.   :0.00000  
 1st Qu.: 23.28   1st Qu.:  46.0   1st Qu.:640.0   1st Qu.:0.09377  
 Median : 41.75   Median : 117.5   Median :654.5   Median :0.12546  
 Mean   : 44.71   Mean   : 303.4   Mean   :654.2   Mean   :0.13593  
 3rd Qu.: 66.86   3rd Qu.: 375.2   3rd Qu.:666.7   3rd Qu.:0.16447  
 Max.   :100.00   Max.   :3324.0   Max.   :706.8   Max.   :0.42083  
                                                                    
    expnstu          str            avginc           elpct       
 Min.   :3926   Min.   :14.00   Min.   : 5.335   Min.   : 0.000  
 1st Qu.:4906   1st Qu.:18.58   1st Qu.:10.639   1st Qu.: 1.941  
 Median :5215   Median :19.72   Median :13.728   Median : 8.778  
 Mean   :5312   Mean   :19.64   Mean   :15.317   Mean   :15.768  
 3rd Qu.:5601   3rd Qu.:20.87   3rd Qu.:17.629   3rd Qu.:22.970  
 Max.   :7712   Max.   :25.80   Max.   :55.328   Max.   :85.540  
                                                                 
    readscr         mathscr     
 Min.   :604.5   Min.   :605.4  
 1st Qu.:640.4   1st Qu.:639.4  
 Median :655.8   Median :652.5  
 Mean   :655.0   Mean   :653.3  
 3rd Qu.:668.7   3rd Qu.:665.9  
 Max.   :704.0   Max.   :709.5  
                                
\end{verbatim}

\hypertarget{question-3}{%
\subsection{Question:}\label{question-3}}

What are the names of the variables in the dataset?

\textbf{A}:

\begin{Shaded}
\begin{Highlighting}[]
\KeywordTok{names}\NormalTok{(Caschool)}
\end{Highlighting}
\end{Shaded}

\begin{verbatim}
 [1] "distcod"  "county"   "district" "grspan"   "enrltot"  "teachers"
 [7] "calwpct"  "mealpct"  "computer" "testscr"  "compstu"  "expnstu" 
[13] "str"      "avginc"   "elpct"    "readscr"  "mathscr" 
\end{verbatim}

\hypertarget{question-4}{%
\subsection{Question:}\label{question-4}}

How many unique observations are available in the variable ``county''

\textbf{A}:

\begin{Shaded}
\begin{Highlighting}[]
\KeywordTok{unique}\NormalTok{(Caschool}\OperatorTok{$}\NormalTok{county)}
\end{Highlighting}
\end{Shaded}

\begin{verbatim}
 [1] Alameda         Butte           Fresno          San Joaquin    
 [5] Kern            Sacramento      Merced          Tulare         
 [9] Los Angeles     Imperial        Monterey        San Diego      
[13] San Bernardino  San Mateo       Ventura         Riverside      
[17] Santa Clara     Madera          Santa Barbara   Orange         
[21] Kings           Sonoma          Contra Costa    Humboldt       
[25] Siskiyou        Lake            Sutter          Mendocino      
[29] San Benito      Shasta          Tehama          Stanislaus     
[33] Tuolumne        El Dorado       Placer          Glenn          
[37] Lassen          Santa Cruz      Nevada          Calaveras      
[41] Marin           San Luis Obispo Inyo            Trinity        
[45] Yuba           
45 Levels: Alameda Butte Calaveras Contra Costa El Dorado Fresno ... Yuba
\end{verbatim}

\hypertarget{question-5}{%
\subsection{Question:}\label{question-5}}

Summarize the mean number of students grouped by county.

\textbf{A}:

\begin{Shaded}
\begin{Highlighting}[]
\NormalTok{mean_countCaschool <-}\StringTok{ }\NormalTok{Caschool }\OperatorTok\StringTok{ }
\StringTok{  }\KeywordTok{group_by}\NormalTok{(county) }\OperatorTok
\StringTok{  }\KeywordTok{summarise}\NormalTok{(}\DataTypeTok{mean_count =} \KeywordTok{mean}\NormalTok{(enrltot)) }\OperatorTok
\StringTok{  }\KeywordTok{arrange}\NormalTok{(}\KeywordTok{desc}\NormalTok{(mean_count))}
\CommentTok{# options(scipen=999) # remove scientific notation}
\CommentTok{# head(mean_countCaschool) #  just show the first six counties}
\end{Highlighting}
\end{Shaded}

\hypertarget{question-6}{%
\subsection{Question:}\label{question-6}}

Calculate the log of average income from of the Caschool dataset. Call
the variable \textbf{logavginc} and add this variable to the dataset.
Then, plot a histogram of the average income vs.~a histogram of log
average income. What do you observe?

\textbf{A}:

\begin{Shaded}
\begin{Highlighting}[]
\NormalTok{Caschool}\OperatorTok{$}\NormalTok{logavginc <-}\StringTok{ }\KeywordTok{log}\NormalTok{(Caschool}\OperatorTok{$}\NormalTok{avginc)}
\end{Highlighting}
\end{Shaded}

\begin{Shaded}
\begin{Highlighting}[]
\KeywordTok{par}\NormalTok{(}\DataTypeTok{mfrow=}\KeywordTok{c}\NormalTok{(}\DecValTok{1}\NormalTok{,}\DecValTok{2}\NormalTok{))  }
\KeywordTok{hist}\NormalTok{(Caschool}\OperatorTok{$}\NormalTok{avginc)}
\KeywordTok{hist}\NormalTok{(Caschool}\OperatorTok{$}\NormalTok{logavginc)}
\end{Highlighting}
\end{Shaded}

\includegraphics{selfeval_1_solutions_files/figure-latex/selfeval_1_solutions-10-1.pdf}

\begin{Shaded}
\begin{Highlighting}[]
\KeywordTok{library}\NormalTok{(gridExtra)}
\end{Highlighting}
\end{Shaded}

\begin{verbatim}

Attaching package: 'gridExtra'
\end{verbatim}

\begin{verbatim}
The following object is masked from 'package:dplyr':

    combine
\end{verbatim}

\begin{Shaded}
\begin{Highlighting}[]
\NormalTok{p1 <-}\StringTok{ }\KeywordTok{ggplot}\NormalTok{(Caschool, }\KeywordTok{aes}\NormalTok{(avginc)) }\OperatorTok{+}
\StringTok{  }\KeywordTok{geom_histogram}\NormalTok{(}\DataTypeTok{show.legend =} \OtherTok{FALSE}\NormalTok{) }\OperatorTok{+}
\StringTok{  }\KeywordTok{theme_bw}\NormalTok{()}
\NormalTok{p2 <-}\StringTok{ }\KeywordTok{ggplot}\NormalTok{(Caschool, }\KeywordTok{aes}\NormalTok{(logavginc)) }\OperatorTok{+}
\StringTok{  }\KeywordTok{geom_histogram}\NormalTok{(}\DataTypeTok{show.legend =} \OtherTok{FALSE}\NormalTok{) }\OperatorTok{+}
\StringTok{  }\KeywordTok{theme_bw}\NormalTok{()}
\KeywordTok{grid.arrange}\NormalTok{(p1, p2,  }
             \DataTypeTok{ncol =} \DecValTok{2}\NormalTok{)  }
\end{Highlighting}
\end{Shaded}

\begin{verbatim}
`stat_bin()` using `bins = 30`. Pick better value with `binwidth`.
\end{verbatim}

\begin{verbatim}
`stat_bin()` using `bins = 30`. Pick better value with `binwidth`.
\end{verbatim}

\includegraphics{selfeval_1_solutions_files/figure-latex/selfeval_1_solutions-11-1.pdf}

Average income is clearly leftward-skewed. The log of averge income
looks more like a normal distribution.

\hypertarget{question-7}{%
\subsection{Question:}\label{question-7}}

We want to create now a subset of counties that have the ten highest
district average income and that have the ten lowest district average
income. Call this subset \emph{Caschool\_lowhighincome}.

\textbf{Hint}: One way is the create two subsets (eg.
Cascholl\_highincome and Caschool\_lowincome and the use the
\texttt{rbind()} function to bind them together.).

\textbf{A}:

\begin{Shaded}
\begin{Highlighting}[]
\NormalTok{Caschool_highincome <-}\StringTok{ }\NormalTok{Caschool }\OperatorTok\StringTok{ }
\KeywordTok{arrange}\NormalTok{(}\KeywordTok{desc}\NormalTok{(avginc)) }\OperatorTok\StringTok{ }
\KeywordTok{head}\NormalTok{(}\DecValTok{10}\NormalTok{)}

\NormalTok{Caschool_lowincome <-}\StringTok{ }\NormalTok{Caschool }\OperatorTok\StringTok{ }
\KeywordTok{arrange}\NormalTok{((avginc)) }\OperatorTok\StringTok{ }
\KeywordTok{head}\NormalTok{(}\DecValTok{10}\NormalTok{)}

\NormalTok{Caschool_lowhighincome <-}\StringTok{ }\KeywordTok{rbind}\NormalTok{(Caschool_highincome,Caschool_highincome)}
\end{Highlighting}
\end{Shaded}

\hypertarget{question-8}{%
\subsection{Question:}\label{question-8}}

Let us test wether a high student/teacher ratio will be associated with
higher-than-average test scores for the school? Create a scatter plot
for the full dataset (\emph{Caschool}) for the variables
\textbf{testscr} and \textbf{str}. You can plot either in base R or use
ggplot2.

\textbf{A}:

Base R-style

\begin{Shaded}
\begin{Highlighting}[]
\KeywordTok{plot}\NormalTok{(}\DataTypeTok{formula =}\NormalTok{ testscr }\OperatorTok{~}\StringTok{ }\NormalTok{str,}
     \DataTypeTok{data =}\NormalTok{ Caschool,}
     \DataTypeTok{xlab =} \StringTok{"Student/Teacher Ratio"}\NormalTok{,}
     \DataTypeTok{ylab =} \StringTok{"Average Test Score"}\NormalTok{, }\DataTypeTok{pch =} \DecValTok{21}\NormalTok{, }\DataTypeTok{col =} \StringTok{'blue'}\NormalTok{)}
\end{Highlighting}
\end{Shaded}

\includegraphics{selfeval_1_solutions_files/figure-latex/selfeval_1_solutions-13-1.pdf}

ggplot2-style

\begin{Shaded}
\begin{Highlighting}[]
\KeywordTok{ggplot}\NormalTok{(}\DataTypeTok{mapping =} \KeywordTok{aes}\NormalTok{(}\DataTypeTok{x =}\NormalTok{ str, }\DataTypeTok{y =}\NormalTok{ testscr), }\DataTypeTok{data =}\NormalTok{ Caschool) }\OperatorTok{+}\StringTok{ }\CommentTok{# base plot}
\StringTok{  }\KeywordTok{geom_point}\NormalTok{() }\OperatorTok{+}\StringTok{ }\CommentTok{# add points}
\StringTok{  }\KeywordTok{scale_y_continuous}\NormalTok{(}\DataTypeTok{name =} \StringTok{"Average Test Score"}\NormalTok{) }\OperatorTok{+}\StringTok{ }
\StringTok{  }\KeywordTok{scale_x_continuous}\NormalTok{(}\DataTypeTok{name =} \StringTok{"Student/Teacher Ratio"}\NormalTok{) }\OperatorTok{+}
\StringTok{  }\KeywordTok{theme_bw}\NormalTok{() }\OperatorTok{+}\StringTok{ }\KeywordTok{ggtitle}\NormalTok{(}\StringTok{"Testscores vs Student/Teacher Ratio"}\NormalTok{)}
\end{Highlighting}
\end{Shaded}

\includegraphics{selfeval_1_solutions_files/figure-latex/selfeval_1_solutions-14-1.pdf}

\hypertarget{question-9}{%
\subsection{Question:}\label{question-9}}

Suppose a policymaker is interested in the following linear model:

\begin{equation}
testscr = \beta_0 +  \beta_1  str + u
\end{equation}

Where \(testscr\) is the average test score for a given school and
\(str\) is the Student/Teacher Ratio (i.e.~the average number of
students per teacher).

Estimate the specified linear model. Is the estimated relationship
between a school's Student/Teacher Ratio and its average test results
postitive or negative?

\textbf{A}:

\begin{Shaded}
\begin{Highlighting}[]
\NormalTok{fit_single <-}\StringTok{ }\KeywordTok{lm}\NormalTok{(}\DataTypeTok{formula =}\NormalTok{ testscr }\OperatorTok{~}\StringTok{ }\NormalTok{str, }\DataTypeTok{data =}\NormalTok{ Caschool)}
\KeywordTok{summary}\NormalTok{(fit_single)}
\end{Highlighting}
\end{Shaded}

\begin{verbatim}

Call:
lm(formula = testscr ~ str, data = Caschool)

Residuals:
    Min      1Q  Median      3Q     Max 
-47.727 -14.251   0.483  12.822  48.540 

Coefficients:
            Estimate Std. Error t value Pr(>|t|)    
(Intercept) 698.9330     9.4675  73.825  < 2e-16 ***
str          -2.2798     0.4798  -4.751 2.78e-06 ***
---
Signif. codes:  0 '***' 0.001 '**' 0.01 '*' 0.05 '.' 0.1 ' ' 1

Residual standard error: 18.58 on 418 degrees of freedom
Multiple R-squared:  0.05124,   Adjusted R-squared:  0.04897 
F-statistic: 22.58 on 1 and 418 DF,  p-value: 2.783e-06
\end{verbatim}

\hypertarget{question-10}{%
\subsection{Question:}\label{question-10}}

Now, plot the regression line for the model we have just estimated.
Again, you can use either base R or ggplot2-style.

\textbf{A}:

Base R-style

\begin{Shaded}
\begin{Highlighting}[]
\NormalTok{fit_california <-}\StringTok{ }\KeywordTok{lm}\NormalTok{(}\DataTypeTok{formula =}\NormalTok{ testscr }\OperatorTok{~}\StringTok{ }\NormalTok{str, }\DataTypeTok{data =}\NormalTok{ Caschool)}
\KeywordTok{plot}\NormalTok{(}\DataTypeTok{formula =}\NormalTok{ testscr }\OperatorTok{~}\StringTok{ }\NormalTok{str,}
     \DataTypeTok{data =}\NormalTok{ Caschool,}
     \DataTypeTok{xlab =} \StringTok{"Student/Teacher Ratio"}\NormalTok{,}
     \DataTypeTok{ylab =} \StringTok{"Average Test Score"}\NormalTok{, }\DataTypeTok{pch =} \DecValTok{21}\NormalTok{, }\DataTypeTok{col =} \StringTok{'blue'}\NormalTok{)}\CommentTok{# same plot as before}
\KeywordTok{abline}\NormalTok{(fit_california, }\DataTypeTok{col =} \StringTok{'red'}\NormalTok{) }\CommentTok{# add regression line}
\end{Highlighting}
\end{Shaded}

\includegraphics{selfeval_1_solutions_files/figure-latex/selfeval_1_solutions-16-1.pdf}

ggplot2-style

\begin{Shaded}
\begin{Highlighting}[]
\KeywordTok{ggplot}\NormalTok{(}\DataTypeTok{mapping =} \KeywordTok{aes}\NormalTok{(}\DataTypeTok{x =}\NormalTok{ str, }\DataTypeTok{y =}\NormalTok{ testscr), }\DataTypeTok{data =}\NormalTok{ Caschool) }\OperatorTok{+}\StringTok{ }\CommentTok{# base plot}
\StringTok{  }\KeywordTok{geom_point}\NormalTok{() }\OperatorTok{+}\StringTok{ }\CommentTok{# add points}
\StringTok{  }\KeywordTok{geom_smooth}\NormalTok{(}\DataTypeTok{method =} \StringTok{"lm"}\NormalTok{, }\DataTypeTok{size=}\DecValTok{1}\NormalTok{, }\DataTypeTok{color=}\StringTok{"red"}\NormalTok{) }\OperatorTok{+}\StringTok{ }\CommentTok{# add regression line}
\StringTok{  }\KeywordTok{scale_y_continuous}\NormalTok{(}\DataTypeTok{name =} \StringTok{"Average Test Score"}\NormalTok{) }\OperatorTok{+}\StringTok{ }
\StringTok{  }\KeywordTok{scale_x_continuous}\NormalTok{(}\DataTypeTok{name =} \StringTok{"Student/Teacher Ratio"}\NormalTok{) }\OperatorTok{+}\StringTok{ }
\StringTok{  }\KeywordTok{theme_bw}\NormalTok{() }\OperatorTok{+}\StringTok{ }\KeywordTok{ggtitle}\NormalTok{(}\StringTok{"Testscores vs Student/Teacher Ratio"}\NormalTok{)}
\end{Highlighting}
\end{Shaded}

\includegraphics{selfeval_1_solutions_files/figure-latex/selfeval_1_solutions-17-1.pdf}

\hypertarget{question-11}{%
\subsection{Question:}\label{question-11}}

Let us extend our example of student test scores by adding families'
average income to our previous model:

\begin{equation}
testscr = \beta_0 +  \beta_1  str +   \beta_2  avginc + u
\end{equation}

\textbf{A}:

\begin{Shaded}
\begin{Highlighting}[]
\NormalTok{fit_multivariate <-}\StringTok{ }\KeywordTok{lm}\NormalTok{(}\DataTypeTok{formula =} \StringTok{"testscr ~ str + avginc"}\NormalTok{, }\DataTypeTok{data =}\NormalTok{ Caschool)}
\KeywordTok{summary}\NormalTok{(fit_multivariate)}
\end{Highlighting}
\end{Shaded}

\begin{verbatim}

Call:
lm(formula = "testscr ~ str + avginc", data = Caschool)

Residuals:
    Min      1Q  Median      3Q     Max 
-39.608  -9.052   0.707   9.259  31.898 

Coefficients:
             Estimate Std. Error t value Pr(>|t|)    
(Intercept) 638.72915    7.44908  85.746   <2e-16 ***
str          -0.64874    0.35440  -1.831   0.0679 .  
avginc        1.83911    0.09279  19.821   <2e-16 ***
---
Signif. codes:  0 '***' 0.001 '**' 0.01 '*' 0.05 '.' 0.1 ' ' 1

Residual standard error: 13.35 on 417 degrees of freedom
Multiple R-squared:  0.5115,    Adjusted R-squared:  0.5091 
F-statistic: 218.3 on 2 and 417 DF,  p-value: < 2.2e-16
\end{verbatim}

Addiing the explanatory variable ``avginc'' to the model, the estimated
coefficient of the student/ teacher ratio becomes first smaller compared
to the previous model and second insignificant at confentional levels.

\hypertarget{question-12}{%
\subsection{Question:}\label{question-12}}

Assume know that ``str'' depends also on the value of yet another
regressor, ``avginc''. Estimate the following model. Compare the sign of
the estimate of \(\beta_2\) and \(\beta_3\). Interpret the results.

\begin{equation}
testscr = \beta_0 +  \beta_1  str +   \beta_2  avginc + \beta_3 (str \times avginc)  + u
\end{equation}

\textbf{A}:

\begin{Shaded}
\begin{Highlighting}[]
\NormalTok{fit_inter =}\StringTok{ }\KeywordTok{lm}\NormalTok{(}\DataTypeTok{formula =}\NormalTok{ testscr }\OperatorTok{~}\StringTok{ }\NormalTok{str }\OperatorTok{+}\StringTok{ }\NormalTok{avginc }\OperatorTok{+}\StringTok{ }\NormalTok{str}\OperatorTok{*}\NormalTok{avginc, }\DataTypeTok{data =}\NormalTok{ Caschool)}
\KeywordTok{summary}\NormalTok{(fit_inter)}
\end{Highlighting}
\end{Shaded}

\begin{verbatim}

Call:
lm(formula = testscr ~ str + avginc + str * avginc, data = Caschool)

Residuals:
    Min      1Q  Median      3Q     Max 
-41.346  -9.260   0.209   8.736  33.368 

Coefficients:
             Estimate Std. Error t value Pr(>|t|)    
(Intercept) 689.47473   14.40894  47.850  < 2e-16 ***
str          -3.40957    0.75980  -4.487 9.34e-06 ***
avginc       -1.62388    0.85214  -1.906   0.0574 .  
str:avginc    0.18988    0.04646   4.087 5.24e-05 ***
---
Signif. codes:  0 '***' 0.001 '**' 0.01 '*' 0.05 '.' 0.1 ' ' 1

Residual standard error: 13.1 on 416 degrees of freedom
Multiple R-squared:  0.5303,    Adjusted R-squared:  0.527 
F-statistic: 156.6 on 3 and 416 DF,  p-value: < 2.2e-16
\end{verbatim}

We observe also that the estimate of \(\beta_2\) changes signs and
becomes negative, while the interaction effect \(\beta_3\) is positive.

This means that an increase in str reduces average student scores (more
students per teacher make it harder to teach effectively); that an
increase in average district income in isolation actually reduces
scores; and that the interaction of both increases scores (more students
per teacher are actually a good thing for student performance in richer
areas).

\hypertarget{question-13}{%
\subsection{Question:}\label{question-13}}

In question 10, 12 and 13, you have fitted 3 models. Report the
regression results, the number of observations, the Akaike information
criterion and the model fit (adj. \(R^2\)) in a formatted table
regression output table. You can use for example the \textbf{stargazer}
or the \textbf{huxtable} package. Which model fits the data best?

\textbf{stargazer} package:

\begin{Shaded}
\begin{Highlighting}[]
\KeywordTok{library}\NormalTok{(stargazer)}
\KeywordTok{invisible}\NormalTok{(}\KeywordTok{stargazer}\NormalTok{(}
\KeywordTok{list}\NormalTok{(fit_single, }
\NormalTok{fit_multivariate,}
\NormalTok{fit_inter)}
\NormalTok{,}\DataTypeTok{keep.stat =} \KeywordTok{c}\NormalTok{(}\StringTok{"n"}\NormalTok{, }\StringTok{"adj.rsq"}\NormalTok{, }\StringTok{"aic"}\NormalTok{, }\StringTok{"bic"}\NormalTok{), }\DataTypeTok{type =} \StringTok{"text"}\NormalTok{, }\DataTypeTok{style =} \StringTok{"ajps"}\NormalTok{))}\CommentTok{# to have number of observations and R^2 reported}
\end{Highlighting}
\end{Shaded}

\begin{verbatim}

-----------------------------------------------
                testscr    testscr       NA    
                Model 1    Model 2    Model 3  
-----------------------------------------------
str            -2.280***   -0.649*   -3.410*** 
                (0.480)    (0.354)    (0.760)  
avginc                     1.839***   -1.624*  
                           (0.093)    (0.852)  
str:avginc                            0.190*** 
                                      (0.046)  
Constant       698.933*** 638.729*** 689.475***
                (9.467)    (7.449)    (14.409) 
N                 420        420        420    
Adj. R-squared   0.049      0.509      0.527   
-----------------------------------------------
***p < .01; **p < .05; *p < .1                 
\end{verbatim}

\textbf{huxtable} package:

\begin{Shaded}
\begin{Highlighting}[]
\KeywordTok{library}\NormalTok{(huxtable)}
\KeywordTok{huxreg}\NormalTok{(fit_single, }
\NormalTok{fit_multivariate, }
\NormalTok{fit_inter, }
\DataTypeTok{statistics =} \KeywordTok{c}\NormalTok{(}\StringTok{"nobs"}\NormalTok{, }\StringTok{"adj.r.squared"}\NormalTok{, }\StringTok{"AIC"}\NormalTok{, }\StringTok{"BIC"}\NormalTok{)) }
\end{Highlighting}
\end{Shaded}

\textbackslash{}begin\{table\}{[}h{]} \centering

\providecommand{\huxb}[2][0,0,0]{\arrayrulecolor[RGB]{#1}\global\arrayrulewidth=#2pt}
    \providecommand{\huxvb}[2][0,0,0]{\color[RGB]{#1}\vrule width #2pt}
    \providecommand{\huxtpad}[1]{\rule{0pt}{\baselineskip+#1}}
    \providecommand{\huxbpad}[1]{\rule[-#1]{0pt}{#1}}
  \begin{tabularx}{0.5\textwidth}{p{0.125\textwidth} p{0.125\textwidth} p{0.125\textwidth} p{0.125\textwidth}}

\textbackslash{}end\{table\}

The adjusted \(R^2\) is highest for the model 3, the model that includes
an interaction term. AIC and BIC, two widely used information criteria,
would also select model 3, relative to each of the other models (The
relatively quality of the model is maximized when the information
criterium is minimized).

\hypertarget{wage1-excercises}{%
\section{Wage1 excercises}\label{wage1-excercises}}

Wooldridge Source: These are data from the 1976 Current Population
Survey.

\begin{Shaded}
\begin{Highlighting}[]
\CommentTok{# install.packages("wooldridge")}
\KeywordTok{library}\NormalTok{(}\StringTok{"wooldridge"}\NormalTok{) }
\KeywordTok{data}\NormalTok{(}\StringTok{"wage1"}\NormalTok{, }\DataTypeTok{package =} \StringTok{"wooldridge"}\NormalTok{)}
\end{Highlighting}
\end{Shaded}

\hypertarget{question-14}{%
\subsection{Question:}\label{question-14}}

Estimate the following model:

\begin{equation}
log(wage) = \beta_0 +  \beta_1 (married \times female) +  \beta_3 educ + \beta_4 exper + beta_5 exper^2 + \beta_6 tenure + \beta_7 tenure^2 + u 
\end{equation}

\begin{enumerate}
\def\labelenumi{\arabic{enumi}.}
\tightlist
\item
  What is the reference group in this model?
\item
  Ceteris paribus, how much more wage do single males make relative to
  the reference group?
\item
  Ceteris paribus, how much more wage do single females make relative to
  the reference group?
\item
  Ceteris paribus, how much less do married females make than single
  females?
\item
  Do the results make sense economically. What socio-economic factors
  could explain the results?
\end{enumerate}

\textbf{A}:

\begin{Shaded}
\begin{Highlighting}[]
\NormalTok{lm2_wage1 <-}\StringTok{ }\KeywordTok{lm}\NormalTok{(}\KeywordTok{log}\NormalTok{(wage)}\OperatorTok{~}\NormalTok{married}\OperatorTok{*}\NormalTok{female}\OperatorTok{+}\NormalTok{educ}\OperatorTok{+}\NormalTok{exper}\OperatorTok{+}\KeywordTok{I}\NormalTok{(exper}\OperatorTok{^}\DecValTok{2}\NormalTok{)}\OperatorTok{+}\NormalTok{tenure}\OperatorTok{+}\KeywordTok{I}\NormalTok{(tenure}\OperatorTok{^}\DecValTok{2}\NormalTok{), }\DataTypeTok{data=}\NormalTok{wage1)}
\KeywordTok{summary}\NormalTok{(lm2_wage1)}
\end{Highlighting}
\end{Shaded}

\begin{verbatim}

Call:
lm(formula = log(wage) ~ married * female + educ + exper + I(exper^2) + 
    tenure + I(tenure^2), data = wage1)

Residuals:
     Min       1Q   Median       3Q      Max 
-1.89697 -0.24060 -0.02689  0.23144  1.09197 

Coefficients:
                 Estimate Std. Error t value Pr(>|t|)    
(Intercept)     0.3213781  0.1000090   3.213 0.001393 ** 
married         0.2126757  0.0553572   3.842 0.000137 ***
female         -0.1103502  0.0557421  -1.980 0.048272 *  
educ            0.0789103  0.0066945  11.787  < 2e-16 ***
exper           0.0268006  0.0052428   5.112 4.50e-07 ***
I(exper^2)     -0.0005352  0.0001104  -4.847 1.66e-06 ***
tenure          0.0290875  0.0067620   4.302 2.03e-05 ***
I(tenure^2)    -0.0005331  0.0002312  -2.306 0.021531 *  
married:female -0.3005931  0.0717669  -4.188 3.30e-05 ***
---
Signif. codes:  0 '***' 0.001 '**' 0.01 '*' 0.05 '.' 0.1 ' ' 1

Residual standard error: 0.3933 on 517 degrees of freedom
Multiple R-squared:  0.4609,    Adjusted R-squared:  0.4525 
F-statistic: 55.25 on 8 and 517 DF,  p-value: < 2.2e-16
\end{verbatim}

\begin{Shaded}
\begin{Highlighting}[]
\KeywordTok{library}\NormalTok{(scales) }\CommentTok{#  quick percent}
\end{Highlighting}
\end{Shaded}

\begin{Shaded}
\begin{Highlighting}[]
\NormalTok{df_lm2_wage1 <-}\StringTok{ }\KeywordTok{tidy}\NormalTok{(lm2_wage1)}
\CommentTok{# Singe male}
\NormalTok{marriedmale <-}\StringTok{ }\NormalTok{df_lm2_wage1 }\OperatorTok
\StringTok{  }\KeywordTok{filter}\NormalTok{(term }\OperatorTok{==}\StringTok{ "married"}\NormalTok{) }\OperatorTok\StringTok{ }
\StringTok{  }\NormalTok{dplyr}\OperatorTok{::}\KeywordTok{select}\NormalTok{(estimate) }\OperatorTok\StringTok{ }
\StringTok{  }\KeywordTok{pull}\NormalTok{() }\CommentTok{# pull out the single coefficient value of the dataframe}
\CommentTok{# Single female}
\NormalTok{singlefemale <-}\StringTok{ }\NormalTok{df_lm2_wage1 }\OperatorTok
\StringTok{  }\KeywordTok{filter}\NormalTok{(term }\OperatorTok{==}\StringTok{ "female"}\NormalTok{) }\OperatorTok\StringTok{ }
\StringTok{  }\NormalTok{dplyr}\OperatorTok{::}\KeywordTok{select}\NormalTok{(estimate) }\OperatorTok\StringTok{ }
\StringTok{  }\KeywordTok{pull}\NormalTok{() }\CommentTok{# pull out the single coefficient value of the dataframe}
\NormalTok{marriedfemale <-}\StringTok{ }\NormalTok{df_lm2_wage1 }\OperatorTok
\StringTok{  }\KeywordTok{filter}\NormalTok{(term }\OperatorTok{==}\StringTok{ "married:female"}\NormalTok{) }\OperatorTok\StringTok{ }
\StringTok{  }\NormalTok{dplyr}\OperatorTok{::}\KeywordTok{select}\NormalTok{(estimate) }\OperatorTok\StringTok{ }
\StringTok{  }\KeywordTok{pull}\NormalTok{() }\CommentTok{# pull out the single coefficient value of the dataframe}
\NormalTok{married<-}\StringTok{ }\NormalTok{df_lm2_wage1 }\OperatorTok
\StringTok{  }\KeywordTok{filter}\NormalTok{(term }\OperatorTok{==}\StringTok{ "married"}\NormalTok{) }\OperatorTok\StringTok{ }\CommentTok{#  }
\StringTok{  }\NormalTok{dplyr}\OperatorTok{::}\KeywordTok{select}\NormalTok{(estimate) }\OperatorTok\StringTok{ }
\StringTok{  }\KeywordTok{pull}\NormalTok{() }\CommentTok{# pull out the single coefficient value of the dataframe}
\end{Highlighting}
\end{Shaded}

\begin{enumerate}
\def\labelenumi{\arabic{enumi}.}
\tightlist
\item
  Reference group: \emph{single} and \emph{male}
\item
  Cp. married males make 21.3\% (\texttt{percent(marriedmale)}) more
  than single males.
\item
  Cp. a single female makes -11.0\% (\texttt{percent(singlefemale)})
  less than the reference group.
\item
  Married females make 8.79\%
  (\texttt{percent(abs(marriedfemale)\ -\ abs(married))}) less than
  single females.
\item
  There seems to be a marriage premium\footnote{There is clearly a
    correlation between men having children and men getting higher
    salaries, and the reverse for women. However, this may reflect the
    fact that women are more likely to withdraw from work to take care
    of children (regardless of whether they'd prefer to), and men may
    double down on work.} for men but for women the marriage premium is
  negative.
\end{enumerate}

\hypertarget{question-15}{%
\subsection{Question:}\label{question-15}}

Test for heteroskedasdicity test in the estimated regression of the
wage1 dataset. Do we reject homoscedasticity for all reasonable
signficance levels? Adjust for heteroscedasticity if necessary by using
refined White heteroscedasticity-robust SE

\begin{Shaded}
\begin{Highlighting}[]
\KeywordTok{bptest}\NormalTok{(lm2_wage1)}
\end{Highlighting}
\end{Shaded}

\begin{verbatim}

    studentized Breusch-Pagan test

data:  lm2_wage1
BP = 13.189, df = 8, p-value = 0.1055
\end{verbatim}

We do not reject the null hypothesis at conventional signficance levels

\hypertarget{question-16}{%
\subsection{Question}\label{question-16}}

\hypertarget{question-17}{%
\subsection{Question}\label{question-17}}

Now, estimate the following model and test again for heteroscedasticity.

\begin{equation}
wage = \beta_0 +  \beta_1  female +  \beta_3 educ + \beta_4 exper + u 
\end{equation}

Adjust for heteroscedasticity if necessary.

\begin{Shaded}
\begin{Highlighting}[]
\NormalTok{lm3_wage1 <-}\StringTok{ }\KeywordTok{lm}\NormalTok{(wage}\OperatorTok{~}\NormalTok{female}\OperatorTok{+}\NormalTok{educ}\OperatorTok{+}\NormalTok{exper, }\DataTypeTok{data=}\NormalTok{wage1)}
\NormalTok{lm3_bptest <-}\StringTok{ }\KeywordTok{bptest}\NormalTok{(lm3_wage1)}
\NormalTok{lm3_bptest}
\end{Highlighting}
\end{Shaded}

\begin{verbatim}

    studentized Breusch-Pagan test

data:  lm3_wage1
BP = 36.904, df = 3, p-value = 4.821e-08
\end{verbatim}

The test statistic of the BP-test is 36.9043336 and the corresponding
p-value is smaller than 4.8208966\times 10\^{}\{-8\}, so we can reject
homoscedasticity for all reasonable signficance levels.

\begin{Shaded}
\begin{Highlighting}[]
\KeywordTok{coeftest}\NormalTok{(lm3_wage1, }\DataTypeTok{vcov=}\NormalTok{hccm)}
\end{Highlighting}
\end{Shaded}

\begin{verbatim}

t test of coefficients:

             Estimate Std. Error t value  Pr(>|t|)    
(Intercept) -1.734481   0.868647 -1.9968   0.04637 *  
female      -2.155517   0.260249 -8.2825 1.020e-15 ***
educ         0.602580   0.065005  9.2697 < 2.2e-16 ***
exper        0.064242   0.010113  6.3521 4.626e-10 ***
---
Signif. codes:  0 '***' 0.001 '**' 0.01 '*' 0.05 '.' 0.1 ' ' 1
\end{verbatim}

\begin{Shaded}
\begin{Highlighting}[]
\NormalTok{cov3 <-}\StringTok{ }\KeywordTok{hccm}\NormalTok{(lm3_wage1, }\DataTypeTok{type=}\StringTok{"hc3"}\NormalTok{) }\CommentTok{# hc3 is the standard method}
\NormalTok{ref.HC3 <-}\StringTok{ }\KeywordTok{coeftest}\NormalTok{(lm3_wage1, }\DataTypeTok{vcov.=}\NormalTok{cov3)}
\NormalTok{ref.HC3}
\end{Highlighting}
\end{Shaded}

\begin{verbatim}

t test of coefficients:

             Estimate Std. Error t value  Pr(>|t|)    
(Intercept) -1.734481   0.868647 -1.9968   0.04637 *  
female      -2.155517   0.260249 -8.2825 1.020e-15 ***
educ         0.602580   0.065005  9.2697 < 2.2e-16 ***
exper        0.064242   0.010113  6.3521 4.626e-10 ***
---
Signif. codes:  0 '***' 0.001 '**' 0.01 '*' 0.05 '.' 0.1 ' ' 1
\end{verbatim}

\hypertarget{collinearity-exercises}{%
\section{Collinearity exercises}\label{collinearity-exercises}}

This exercise focuses on the \textbf{collineartiy} problem.

\hypertarget{question-18}{%
\subsection{Question:}\label{question-18}}

Perform the following commands in R:

\begin{Shaded}
\begin{Highlighting}[]
\KeywordTok{set.seed}\NormalTok{(}\DecValTok{1}\NormalTok{)}
\NormalTok{x1 <-}\StringTok{ }\KeywordTok{runif}\NormalTok{(}\DecValTok{100}\NormalTok{)}
\NormalTok{x2 <-}\StringTok{ }\FloatTok{0.5} \OperatorTok{*}\StringTok{ }\NormalTok{x1 }\OperatorTok{+}\StringTok{ }\KeywordTok{rnorm}\NormalTok{(}\DecValTok{100}\NormalTok{)}\OperatorTok{/}\DecValTok{10}
\NormalTok{y <-}\StringTok{ }\DecValTok{2} \OperatorTok{+}\DecValTok{2}\OperatorTok{*}\NormalTok{x1 }\OperatorTok{+}\StringTok{ }\FloatTok{0.3} \OperatorTok{*}\NormalTok{x2 }\OperatorTok{+}\KeywordTok{rnorm}\NormalTok{(}\DecValTok{100}\NormalTok{)}
\end{Highlighting}
\end{Shaded}

The last line corresponds to creating a linear model in which \(y\) is a
function of \(x_1\) and \(x_2\). Write out the form of the linear model.
What are the regression coefficients?

\textbf{A}:

\(y = 2 +2x_1 + 0.3x_2 + \epsilon\)

\(\beta_0 = 2\), \(\beta_1 = 2\), \(\beta_3 = 0.3\)

\hypertarget{question-19}{%
\subsection{Question:}\label{question-19}}

What is the correlation between \(x_1\) and \(x_2\)? Create a
scatterplot displaying the relationship between the variables.

\textbf{A}:

\begin{Shaded}
\begin{Highlighting}[]
\KeywordTok{cor}\NormalTok{(x1, x2)}
\end{Highlighting}
\end{Shaded}

\begin{verbatim}
[1] 0.8351212
\end{verbatim}

Base R style:

\begin{Shaded}
\begin{Highlighting}[]
\KeywordTok{plot}\NormalTok{(x1, x2)}
\end{Highlighting}
\end{Shaded}

\includegraphics{selfeval_1_solutions_files/figure-latex/selfeval_1_solutions-31-1.pdf}

ggplot2 style:

\begin{Shaded}
\begin{Highlighting}[]
\NormalTok{d <-}\StringTok{  }\KeywordTok{data.frame}\NormalTok{(x1,x2)}
\KeywordTok{ggplot}\NormalTok{(d, }\KeywordTok{aes}\NormalTok{(x1, x2)) }\OperatorTok{+}
\StringTok{  }\KeywordTok{geom_point}\NormalTok{(}\DataTypeTok{shape =} \DecValTok{16}\NormalTok{, }\DataTypeTok{size =} \DecValTok{3}\NormalTok{, }\DataTypeTok{show.legend =} \OtherTok{FALSE}\NormalTok{) }\OperatorTok{+}
\StringTok{  }\KeywordTok{theme_minimal}\NormalTok{() }
\end{Highlighting}
\end{Shaded}

\includegraphics{selfeval_1_solutions_files/figure-latex/selfeval_1_solutions-32-1.pdf}

\hypertarget{question-20}{%
\subsection{Question:}\label{question-20}}

Using this data, fit a least squares regression to predict \(y\) using
\(x_1\) and \(x_2\). Describe the results obtained. What are
\(\hat{\beta_0}\), \(\hat{\beta_1}\) and \(\hat{\beta_2}\)? How do these
relate to the true \(\beta_0\), \(\beta_1\) and \(\beta_2\)? Can you
reject the null hypothesis \(H_0: \beta_1 =0\)? How about the null
hypothesis \(H_0: \beta_2 =0\)?

\textbf{A}:

\begin{Shaded}
\begin{Highlighting}[]
\NormalTok{lm.fit =}\StringTok{ }\KeywordTok{lm}\NormalTok{(y}\OperatorTok{~}\NormalTok{x1}\OperatorTok{+}\NormalTok{x2)}
\KeywordTok{summary}\NormalTok{(lm.fit)}
\end{Highlighting}
\end{Shaded}

\begin{verbatim}

Call:
lm(formula = y ~ x1 + x2)

Residuals:
    Min      1Q  Median      3Q     Max 
-2.8311 -0.7273 -0.0537  0.6338  2.3359 

Coefficients:
            Estimate Std. Error t value Pr(>|t|)    
(Intercept)   2.1305     0.2319   9.188 7.61e-15 ***
x1            1.4396     0.7212   1.996   0.0487 *  
x2            1.0097     1.1337   0.891   0.3754    
---
Signif. codes:  0 '***' 0.001 '**' 0.01 '*' 0.05 '.' 0.1 ' ' 1

Residual standard error: 1.056 on 97 degrees of freedom
Multiple R-squared:  0.2088,    Adjusted R-squared:  0.1925 
F-statistic:  12.8 on 2 and 97 DF,  p-value: 1.164e-05
\end{verbatim}

The regression coefficients are close to the true coefficients, although
with high standard error. We can reject the null hypothesis for
\(\beta_1\) because its p-value is below 5\%. We cannot reject the null
hypothesis for \(\beta_2\) because its p-value is much above the 5\%
typical cutoff, over 60\%.

\hypertarget{question-21}{%
\subsection{Question:}\label{question-21}}

Now fit least squares regression to predict \(y\) using only \(x_1\).
Comment on your results. Can you reject the null hypothesis
\(H_0: \beta_1=0\)?

\textbf{A}:

\begin{Shaded}
\begin{Highlighting}[]
\NormalTok{lm.fit =}\StringTok{ }\KeywordTok{lm}\NormalTok{(y}\OperatorTok{~}\NormalTok{x1)}
\KeywordTok{summary}\NormalTok{(lm.fit)}
\end{Highlighting}
\end{Shaded}

\begin{verbatim}

Call:
lm(formula = y ~ x1)

Residuals:
     Min       1Q   Median       3Q      Max 
-2.89495 -0.66874 -0.07785  0.59221  2.45560 

Coefficients:
            Estimate Std. Error t value Pr(>|t|)    
(Intercept)   2.1124     0.2307   9.155 8.27e-15 ***
x1            1.9759     0.3963   4.986 2.66e-06 ***
---
Signif. codes:  0 '***' 0.001 '**' 0.01 '*' 0.05 '.' 0.1 ' ' 1

Residual standard error: 1.055 on 98 degrees of freedom
Multiple R-squared:  0.2024,    Adjusted R-squared:  0.1942 
F-statistic: 24.86 on 1 and 98 DF,  p-value: 2.661e-06
\end{verbatim}

Yes, we can reject the null hypothesis for the regression coefficient
given the p-value for its t-statistic is near zero.

\hypertarget{question-22}{%
\subsection{Question:}\label{question-22}}

Now fit least squares regression to predict \(y\) using only \(x_2\).
Comment on your results. Can you reject the null hypothesis
\(H_0: \beta_2=0\)?

\textbf{A}:

\begin{Shaded}
\begin{Highlighting}[]
\NormalTok{lm.fit =}\StringTok{ }\KeywordTok{lm}\NormalTok{(y}\OperatorTok{~}\NormalTok{x2)}
\KeywordTok{summary}\NormalTok{(lm.fit)}
\end{Highlighting}
\end{Shaded}

\begin{verbatim}

Call:
lm(formula = y ~ x2)

Residuals:
     Min       1Q   Median       3Q      Max 
-2.62687 -0.75156 -0.03598  0.72383  2.44890 

Coefficients:
            Estimate Std. Error t value Pr(>|t|)    
(Intercept)   2.3899     0.1949   12.26  < 2e-16 ***
x2            2.8996     0.6330    4.58 1.37e-05 ***
---
Signif. codes:  0 '***' 0.001 '**' 0.01 '*' 0.05 '.' 0.1 ' ' 1

Residual standard error: 1.072 on 98 degrees of freedom
Multiple R-squared:  0.1763,    Adjusted R-squared:  0.1679 
F-statistic: 20.98 on 1 and 98 DF,  p-value: 1.366e-05
\end{verbatim}

Yes, we can reject the null hypothesis for the regression coefficient
given the p-value for its t-statistic is near zero.

\hypertarget{question-23}{%
\subsection{Question:}\label{question-23}}

Do the results from the previous questions contradict each other?
Explain your answer.

\textbf{A}:

No, because \(x_1\) and \(x_2\) have collinearity, it is hard to
distinguish their effects when regressed upon together. When they are
regressed upon separately, the linear relationship between y and each
predictor is indicated more clearly.

\hypertarget{question-24}{%
\subsection{Question:}\label{question-24}}

Now suppose we obtain one additional observation, which was
unfortunately mismeasured.

\begin{Shaded}
\begin{Highlighting}[]
\NormalTok{x1 <-}\StringTok{ }\KeywordTok{c}\NormalTok{(x1, }\FloatTok{0.1}\NormalTok{)}
\NormalTok{x2 <-}\StringTok{ }\KeywordTok{c}\NormalTok{(x2, }\FloatTok{0.8}\NormalTok{)}
\NormalTok{y =}\StringTok{ }\KeywordTok{c}\NormalTok{(y,}\DecValTok{6}\NormalTok{)}
\end{Highlighting}
\end{Shaded}

Re-fit the linear model using the new data. What effect does this new
observation have on the each of the models? In each model, is this
observation an outlier?

\textbf{A}:

\begin{Shaded}
\begin{Highlighting}[]
\NormalTok{lm.fit1 =}\StringTok{ }\KeywordTok{lm}\NormalTok{(y}\OperatorTok{~}\NormalTok{x1}\OperatorTok{+}\NormalTok{x2)}
\KeywordTok{summary}\NormalTok{(lm.fit1)}
\end{Highlighting}
\end{Shaded}

\begin{verbatim}

Call:
lm(formula = y ~ x1 + x2)

Residuals:
     Min       1Q   Median       3Q      Max 
-2.73348 -0.69318 -0.05263  0.66385  2.30619 

Coefficients:
            Estimate Std. Error t value Pr(>|t|)    
(Intercept)   2.2267     0.2314   9.624 7.91e-16 ***
x1            0.5394     0.5922   0.911  0.36458    
x2            2.5146     0.8977   2.801  0.00614 ** 
---
Signif. codes:  0 '***' 0.001 '**' 0.01 '*' 0.05 '.' 0.1 ' ' 1

Residual standard error: 1.075 on 98 degrees of freedom
Multiple R-squared:  0.2188,    Adjusted R-squared:  0.2029 
F-statistic: 13.72 on 2 and 98 DF,  p-value: 5.564e-06
\end{verbatim}

\begin{Shaded}
\begin{Highlighting}[]
\NormalTok{lm.fit2 =}\StringTok{ }\KeywordTok{lm}\NormalTok{(y}\OperatorTok{~}\NormalTok{x1)}
\KeywordTok{summary}\NormalTok{(lm.fit2)}
\end{Highlighting}
\end{Shaded}

\begin{verbatim}

Call:
lm(formula = y ~ x1)

Residuals:
    Min      1Q  Median      3Q     Max 
-2.8897 -0.6556 -0.0909  0.5682  3.5665 

Coefficients:
            Estimate Std. Error t value Pr(>|t|)    
(Intercept)   2.2569     0.2390   9.445 1.78e-15 ***
x1            1.7657     0.4124   4.282 4.29e-05 ***
---
Signif. codes:  0 '***' 0.001 '**' 0.01 '*' 0.05 '.' 0.1 ' ' 1

Residual standard error: 1.111 on 99 degrees of freedom
Multiple R-squared:  0.1562,    Adjusted R-squared:  0.1477 
F-statistic: 18.33 on 1 and 99 DF,  p-value: 4.295e-05
\end{verbatim}

\begin{Shaded}
\begin{Highlighting}[]
\NormalTok{lm.fit3 =}\StringTok{ }\KeywordTok{lm}\NormalTok{(y}\OperatorTok{~}\NormalTok{x2)}
\KeywordTok{summary}\NormalTok{(lm.fit3)}
\end{Highlighting}
\end{Shaded}

\begin{verbatim}

Call:
lm(formula = y ~ x2)

Residuals:
     Min       1Q   Median       3Q      Max 
-2.64729 -0.71021 -0.06899  0.72699  2.38074 

Coefficients:
            Estimate Std. Error t value Pr(>|t|)    
(Intercept)   2.3451     0.1912  12.264  < 2e-16 ***
x2            3.1190     0.6040   5.164 1.25e-06 ***
---
Signif. codes:  0 '***' 0.001 '**' 0.01 '*' 0.05 '.' 0.1 ' ' 1

Residual standard error: 1.074 on 99 degrees of freedom
Multiple R-squared:  0.2122,    Adjusted R-squared:  0.2042 
F-statistic: 26.66 on 1 and 99 DF,  p-value: 1.253e-06
\end{verbatim}

In the first model, it shifts \(x_1\) to statistically insignificance
and shifts \(x_2\) to statistiscal significance from the change in
p-values between the two linear regressions.

\begin{Shaded}
\begin{Highlighting}[]
\KeywordTok{library}\NormalTok{(ggfortify)}
\end{Highlighting}
\end{Shaded}

\begin{Shaded}
\begin{Highlighting}[]
\KeywordTok{autoplot}\NormalTok{(lm.fit1)}
\end{Highlighting}
\end{Shaded}

\includegraphics{selfeval_1_solutions_files/figure-latex/selfeval_1_solutions-41-1.pdf}

\begin{Shaded}
\begin{Highlighting}[]
\KeywordTok{autoplot}\NormalTok{(lm.fit2)}
\end{Highlighting}
\end{Shaded}

\includegraphics{selfeval_1_solutions_files/figure-latex/selfeval_1_solutions-42-1.pdf}

\begin{Shaded}
\begin{Highlighting}[]
\KeywordTok{autoplot}\NormalTok{(lm.fit3)}
\end{Highlighting}
\end{Shaded}

\includegraphics{selfeval_1_solutions_files/figure-latex/selfeval_1_solutions-43-1.pdf}

The additional observation for \(x_2\) seems to become a high leverage
point.


\end{document}
